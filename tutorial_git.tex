\documentclass[11pt]{article}
\pagestyle{myheadings}

\usepackage{graphicx}
\usepackage[german]{babel}
%\usepackage{helvetic}
\usepackage{a4wide}
\usepackage{amssymb}
\usepackage[clearempty]{titlesec}
\usepackage[bf]{caption2} 
\usepackage{hyperref}

\parskip3mm
\parindent0mm

\begin{document}
%\vspace*{-1cm}
  {\normalsize  
Severin Reiz\hfill MPE, March 2019\\ 
    \mbox{}\\ [2ex] }
%\vspace*{-1.5cm}
%\begin{center}
%\parbox{16cm}{
%  \includegraphics{../Bilder/logo.ps}
%}
%\end{center}

%\maketitle \vspace{-8ex}
%\vspace{-1.5cm}
\thispagestyle{empty}
\begin{center}
    {\LARGE\sffamily\bfseries Git course - Tutorial}
\end{center}


\section*{Setting up git}
This task is simply meant to let you set up everything to be able to use git. 
\begin{itemize}
\item[\bf{a)}] Install git and a client if you like. Instuctions available at course website (See end of sheet).%https://www5.in.tum.de/wiki/index.php/Python_/_Git_/_Bash_Course}{https://www5.in.tum.de/}
\item[\bf{b)}] Get yourself an account at bitbucket\footnote{\url{https://bitbucket.org}} or github\footnote{\url{https://github.com}} or some similar platform. Maybe you already have access to some GitLab server? Since Microsoft acquired github, github also offers private repositories for free accounts.
\item[\bf{c)}] Use the course's git repo:
\begin{itemize}
\item Clone the repository: (on terminal git clone )\\ 
HTML: https://github.com/severin617/IMPRS-course-2019.git \\
SSH: git@github.com:severin617/IMPRS-course-2019.git
\item Compile Latex files, etc.
\end{itemize}
\item[\bf{d)}] Create your own new repository (remote on github/gitlab/bitbucket or local).
\begin{itemize}
\item Clone the repository to your machine.
\item Copy things into your local version of the repository. Maybe you want to write a new paper soon? Maybe some code?
\item Create your first commit and push it.
\end{itemize}
\end{itemize}

\section*{Optional: Branches}
If you have code, latex or anything: Set a github repository for it. It's best to learn from own examples. If you decide to later get rid of it, "OK!", but you gave it a try!
 
I don't know your background and how fast you are in grasping new things: In collaborative research projects, git branches are unavoidable.
\begin{itemize}
\item Make a local branch, for example in it@github.com:severin617/IMPRS-course-2019.git
\begin{itemize}
\item[] git checkout -b [branch-name (your name?)] \# creates a branch
\item[] git branch \# shows what branch you're on 
\end{itemize}
\item Push the branch to github
\begin{itemize}
\item[] git push origin [branch-name (your name?)]
\end{itemize}
\item Check the github webpage to see if it worked!
\end{itemize}



\section*{Website}
\label{sec:Website}
The website with slides and tutorials can be found at\\
\url{https://www5.in.tum.de/wiki/index.php/Python_/_Git_/_Bash_Course}.

\end{document}

